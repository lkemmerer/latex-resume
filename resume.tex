%% start of file `template.tex'.
%% Copyright 2006-2013 Xavier Danaux (xdanaux@gmail.com).
%
% This work may be distributed and/or modified under the
% conditions of the LaTeX Project Public License version 1.3c,
% available at http://www.latex-project.org/lppl/.


\documentclass[11pt,a4paper,sans]{moderncv}        % possible options include font size ('10pt', '11pt' and '12pt'), paper size ('a4paper', 'letterpaper', 'a5paper', 'legalpaper', 'executivepaper' and 'landscape') and font family ('sans' and 'roman')

% moderncv themes
\moderncvstyle{classic}                             % style options are 'casual' (default), 'classic', 'oldstyle' and 'banking'
\moderncvcolor{purple}                               % color options 'blue' (default), 'orange', 'green', 'red', 'purple', 'grey' and 'black'
%\renewcommand{\familydefault}{\sfdefault}         % to set the default font; use '\sfdefault' for the default sans serif font, '\rmdefault' for the default roman one, or any tex font name
%\nopagenumbers{}                                  % uncomment to suppress automatic page numbering for CVs longer than one page

% character encoding
\usepackage[utf8]{inputenc}                       % if you are not using xelatex ou lualatex, replace by the encoding you are using
%\usepackage{CJKutf8}                              % if you need to use CJK to typeset your resume in Chinese, Japanese or Korean

% adjust the page margins
\usepackage[scale=0.75]{geometry}
%\setlength{\hintscolumnwidth}{3cm}                % if you want to change the width of the column with the dates
%\setlength{\makecvtitlenamewidth}{10cm}           % for the 'classic' style, if you want to force the width allocated to your name and avoid line breaks. be careful though, the length is normally calculated to avoid any overlap with your personal info; use this at your own typographical risks...

% personal data
\name{Laurie}{Kemmerer}
%\title{Resumé title}                               % optional, remove / comment the line if not wanted
%\address{street and number}{postcode city}{country}% optional, remove / comment the line if not wanted; the "postcode city" and and "country" arguments can be omitted or provided empty
\phone[mobile]{+1~(503)~866~4437}                   % optional, remove / comment the line if not wanted
\email{laurie.kemmerer@gmail.com}                               % optional, remove / comment the line if not wanted
\homepage{http://github.com/lkemmerer}                         % optional, remove / comment the line if not wanted
%\extrainfo{additional information}                 % optional, remove / comment the line if not wanted
%\photo[64pt][0.4pt]{picture}                       % optional, remove / comment the line if not wanted; '64pt' is the height the picture must be resized to, 0.4pt is the thickness of the frame around it (put it to 0pt for no frame) and 'picture' is the name of the picture file
%\quote{Some quote}                                 % optional, remove / comment the line if not wanted

% to show numerical labels in the bibliography (default is to show no labels); only useful if you make citations in your resume
%\makeatletter
%\renewcommand*{\bibliographyitemlabel}{\@biblabel{\arabic{enumiv}}}
%\makeatother
%\renewcommand*{\bibliographyitemlabel}{[\arabic{enumiv}]}% CONSIDER REPLACING THE ABOVE BY THIS

% bibliography with mutiple entries
%\usepackage{multibib}
%\newcites{book,misc}{{Books},{Others}}
%----------------------------------------------------------------------------------
%            content
%----------------------------------------------------------------------------------
\begin{document}
%\begin{CJK*}{UTF8}{gbsn}                          % to typeset your resume in Chinese using CJK
%-----       resume       ---------------------------------------------------------
\makecvtitle

\section{Languages}
\cvitemwithcomment{Ruby}{Advanced}{...my favorite language}
\cvitemwithcomment{JavaScript}{Intermediate}{...always a peripheral language}
\cvitemwithcomment{C\#}{Intermediate}{...it's OK. Still feeling rusty}
\cvitemwithcomment{Scala}{Beginner/Intermediate}{...pretending it's a statically typed cross between Ruby and JavaScript and hoping nobody notices}

\section{Testing Tools}
\cvitem{Ruby}{RSpec, Capybara, Cucumber, Minitest}
\cvdoubleitem{JavaScript}{Jasmine, Karma}{Browser}{Selenium, Watir, PhantomJS}
\cvdoubleitem{C\#}{NUnit, Moq, XUnit, Specflow}{Scala}{Scalatest, Scalamock}

\section{Other Technologies}
\cvdoubleitem{Databases}{PostgreSQL, MongoDB, MS SQL Server}{Web}{Ruby on Rails, Sinatra, Akka HTTP, AngularJS, ServiceStack, BackboneJS}
\cvdoubleitem{CI}{TravisCI, TeamCity, Jenkins}{Devopsish}{Terraform, Docker, Vagrant}
\cvitem{etc}{Akka, RabbitMQ, AWS (EC2, SQS, SNS, S3...)}

\section{Experience}
\cventry{Jan 2016 to present}{Software Engineer/Team Lead}{ShiftWise}{Portland, OR}{}{I transferred to the Platform team where I worked as an application engineer to create the infrastructure for an event-driven application ecosystem in Scala with Akka that will be used for high-volume data processing as well as feed into our reporting system. I continued to provide testing guidance and was, for better or (more probably) for worse, considered a resource for making our code more in line with Scala idioms.\newline{}\newline{}In February, I was made team lead. In this role I facilitated communication with management, other teams, and stakeholders and work as an advocate for my team and their needs.}
\cventry{Jan 2016 to Sept 2016}{Software Development Engineer in Test}{ShiftWise}{Portland, OR}{}{I helped build and maintain testing and CI infrastructure for the next generation vendor management system. I also devoted much of my time to formulating and implementing testing strategy with my fellow SDETs, reviewing code and tests for my sprint teams, guiding design decisions to make our system more testable and maintainable. Since all the SDETs were on multiple sprint teams, we were able to facilitate communication between teams around new and evolving design patterns, lessons learned, and possible risks in unfamiliar areas of the code base.}
\cventry{Aug 2015 to Jan 2016}{Remote Software Engineer/Contractor}{Zivity}{San Francisco, CA}{}{Using Rails and Backbone, I implemented features for a crowdfunding service that integrates with their existing site.}
\cventry{Feb to May 2015}{Technical Consultant}{Renew Financial}{Portland, OR}{}{I initially worked with the product team to add new features using AngularJS and Ruby. Due to looming deadlines, the QA team asked that I move to their team full time.\newline {
I pair-programmed with QA, mentoring in Ruby and OOP skills. I also improved their home-spun framework, making tests more maintainable and less error-prone.}
}
\cventry{May 2014 to Jan 2015}{Software Engineer}{CrowdCompass}{Portland, OR}{}{As part of a cross-functional agile team, I collaborated to deliver a one-to-one messaging feature. I worked with devops to foster a more service-oriented architecture. I also functioned as the liaison between my team and the 3rd party whose service we leveraged.
\newline{}\newline {
I maintained and added features to our monolithic Rails apps. In doing so, I increased test coverage and refactored in order to simplify future work to separate functionality into independent services.}
\newline{I also instituted daily standup for the server team and pushed for mandatory code reviews. This helped bring about an environment more conducive to knowledge sharing and mentorship.}
}
\cventry{May 2013 to May 2014}{Senior Software Engineer}{Reneable Funding (now Renew Financial)}{Portland, OR}{}{I designed and implemented well-factored, service-oriented (Ruby-beyond-Rails) green-energy-financing applications and loan processing software. I maintained existing projects and contributed to a page-object-model framework for use with Cucumber.
}
\cventry{Mar 2010 to May 2013}{Software Developer}{Renewable Funding (now Renew Financial)}{Portland, OR}{}{
Using BDD and TDD, I developed features for a multi-tenant Rails application. This app provided homeowners with energy efficiency information and connections to contractors and financing. I also contributed to GIS (postgis) functionality that enabled property-specific savings calculations.
\newline{
I standardized energy retrofit data from local governments and utilities and used the results to created a data warehouse for reporting purposes.}
}
\cventry{Nov 2007 to Mar 2010}{Software Engineer}{Harland Financial Solutions (now D+H)}{Portland, OR}{}{
 gathered requirements for internal applications that increased user efficiency and revenue recognition. I designed, developed, and maintained these apps using ASP.NET MVC, Web Forms, and Winforms. I advocated for code reviews and other process improvements.
}
\cventry{Apr to Nov 2007}{QA Engineer}{Renew Financial}{Portland, OR}{}{
 I gathered requirements for internal applications that increased user efficiency and revenue recognition. I designed, developed, and maintained these apps using ASP.NET MVC, Web Forms, and Winforms. I advocated for code reviews and other process improvements.
}
\cventry{Apr 2006 to Jan 2007}{Lead QA Techician/defacto Customer Support}{Instant Media}{Portland, OR}{}{
 I was responsible for giving final sign-off on builds and database changes. I maintained test documentation and plans for our internet TV/RSS feed Windows application. I performed ad-hoc, scenario, and usability testing.
 \newline {
I managed outsourced technical support representatives and integrated feedback into design requirements.}
}

\section{Education}
\cventry{2001--2006}{B.S. Computer Engineering}{Penn State University}{University Park, PA}{\textit{Grade}}{In addition to gaining a degree, I worked as Lead Technology Tutor for a program that tutored faculty in how to use computers and managed tutor scheduling and assignment.  I also spent a summer working as a research assistant in the Human Computer Interaction lab. During that time, I participated in organizing a conference to help foster communication between non-profits about how they use technology.  }  % arguments 3 to 6 can be left empty

\clearpage
%\clearpage\end{CJK*}                              % if you are typesetting your resume in Chinese using CJK; the \clearpage is required for fancyhdr to work correctly with CJK, though it kills the page numbering by making \lastpage undefined
\end{document}


%% end of file `template.tex'.

